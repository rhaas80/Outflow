% *======================================================================*
%  Cactus Thorn template for ThornGuide documentation
%  Author: Ian Kelley
%  Date: Sun Jun 02, 2002
%  $Header: /cactusdevcvs/Cactus/doc/ThornGuide/template.tex,v 1.12 2004/01/07 20:12:39 rideout Exp $
%
%  Thorn documentation in the latex file doc/documentation.tex
%  will be included in ThornGuides built with the Cactus make system.
%  The scripts employed by the make system automatically include
%  pages about variables, parameters and scheduling parsed from the
%  relevant thorn CCL files.
%
%  This template contains guidelines which help to assure that your
%  documentation will be correctly added to ThornGuides. More
%  information is available in the Cactus UsersGuide.
%
%  Guidelines:
%   - Do not change anything before the line
%       % START CACTUS THORNGUIDE",
%     except for filling in the title, author, date, etc. fields.
%        - Each of these fields should only be on ONE line.
%        - Author names should be separated with a \\ or a comma.
%   - You can define your own macros, but they must appear after
%     the START CACTUS THORNGUIDE line, and must not redefine standard
%     latex commands.
%   - To avoid name clashes with other thorns, 'labels', 'citations',
%     'references', and 'image' names should conform to the following
%     convention:
%       ARRANGEMENT_THORN_LABEL
%     For example, an image wave.eps in the arrangement CactusWave and
%     thorn WaveToyC should be renamed to CactusWave_WaveToyC_wave.eps
%   - Graphics should only be included using the graphicx package.
%     More specifically, with the "\includegraphics" command.  Do
%     not specify any graphic file extensions in your .tex file. This
%     will allow us to create a PDF version of the ThornGuide
%     via pdflatex.
%   - References should be included with the latex "\bibitem" command.
%   - Use \begin{abstract}...\end{abstract} instead of \abstract{...}
%   - Do not use \appendix, instead include any appendices you need as
%     standard sections.
%   - For the benefit of our Perl scripts, and for future extensions,
%     please use simple latex.
%
% *======================================================================*
%
% Example of including a graphic image:
%    \begin{figure}[ht]
% 	\begin{center}
%    	   \includegraphics[width=6cm]{MyArrangement_MyThorn_MyFigure}
% 	\end{center}
% 	\caption{Illustration of this and that}
% 	\label{MyArrangement_MyThorn_MyLabel}
%    \end{figure}
%
% Example of using a label:
%   \label{MyArrangement_MyThorn_MyLabel}
%
% Example of a citation:
%    \cite{MyArrangement_MyThorn_Author99}
%
% Example of including a reference
%   \bibitem{MyArrangement_MyThorn_Author99}
%   {J. Author, {\em The Title of the Book, Journal, or periodical}, 1 (1999),
%   1--16. {\tt http://www.nowhere.com/}}
%
% *======================================================================*

% If you are using CVS use this line to give version information
% $Header: /cactusdevcvs/Cactus/doc/ThornGuide/template.tex,v 1.12 2004/01/07 20:12:39 rideout Exp $

\documentclass{article}

% Use the Cactus ThornGuide style file
% (Automatically used from Cactus distribution, if you have a
%  thorn without the Cactus Flesh download this from the Cactus
%  homepage at www.cactuscode.org)
\usepackage{../../../../doc/latex/cactus}

\begin{document}

% The author of the documentation
\author{Roland Haas \textless roland.haas@physcis.gatech.edu\textgreater}

% The title of the document (not necessarily the name of the Thorn)
\title{Outflow}

% the date your document was last changed, if your document is in CVS,
% please use:
%    \date{$ $Date: 2004/01/07 20:12:39 $ $}
\date{August 15 2009}

\maketitle

% Do not delete next line
% START CACTUS THORNGUIDE

% Add all definitions used in this documentation here
%   \def\mydef etc

% Add an abstract for this thorn's documentation
\begin{abstract}
    Outflow calculates the flow of rest mass density  across a
    SphericalSurface, eg. and apparent horizon or a sphere at ``infinity''.
\end{abstract}

% The following sections are suggestive only.
% Remove them or add your own.

\section{Introduction}
Hydrodynamic simulations should conserve eg. the total rest mass in the system
(outside he event horizon, that is). This thorn allow to measure the flux of
rest mass across a given SphericalSurface.

\section{Physical System}
The Valencia formalism defines its $D = \rho W$ variable such that one can define a 
total rest mass density as:
\begin{displaymath}
    M = \int \sqrt{\gamma} D d^3x
\end{displaymath}
and from the EOM of $D$ (see~\cite{lrr-font})
\begin{displaymath}
    \frac{\partial }{\partial x^0} (\sqrt{\gamma}D )+ 
    \frac{\partial }{\partial x^i} (\sqrt(\gamma) \alpha D (v^i- 
      \beta^i/\alpha) = 0
\end{displaymath}
one obtains:
\begin{eqnarray}
\dot M
&=& - \int_V \frac{\partial }{\partial x^i} 
   (\sqrt\gamma \alpha D(v^i - \beta^i/\alpha) d^3x \\
&=& -\int_{\partial V} \sqrt\gamma \alpha D (v^i-\beta^i/\alpha) d\sigma_i
\end{eqnarray}
where $\sigma_i$ is the ordinary flat space directed surface element of 
the enclosing surface, eg.\
\begin{displaymath}
d\sigma_i = \hat r_i r^2 \sin\theta d\theta d\phi
\end{displaymath}
with $\hat r_i = [\cos\phi\sin\theta, \sin\phi\sin\theta,
\cos(\theta)]$ 
for a sphere of radius r.

For a generic SphericalSurface parametrized by $\theta$ and $\phi$ one has:
\begin{eqnarray}
    x &=& \bar r(\theta, \phi) \cos\phi \sin\theta \\
    y &=& \bar r(\theta, \phi) \sin\phi \sin\theta \\
    z &=& \bar r(\theta, \phi)          \cos\theta
\end{eqnarray}
where $\bar r$ is the isotropic radius.
Consequently the surface element is
\begin{eqnarray}
    d \sigma_i 
    &=& \left( 
        \frac{\partial\vec{\bar r}}{\partial\theta} \times
        \frac{\partial\vec{\bar r}}{\partial\phi}
        \right)_i \\
    &=& \bar r^2 \sin\theta \hat{\bar r}_i 
        - \frac{\partial\bar r}{\partial\theta} \bar r \sin \theta \hat \theta_i
        - \frac{\partial\bar r}{\partial\phi} \bar r \hat \phi_i
\end{eqnarray}
where $\hat{\bar r}$, $\hat \theta$ and $\hat \phi$ are the flat space
standard unit vectors on the sphere~\cite{mtw:73}. 

\section{Numerical Implementation}
We implement the surface integral by interpolating the required quantities
($g_{ij}$, $\rho$, $v^i$, $\beta^i$, $\alpha$) onto the spherical surface and
then integrate using a fourth order convergent Newton-Cotes formula. 

For the
$\theta$ direction SphericalSurfaces defines grid points such that
\begin{displaymath}
    \theta_i = -(n_\theta - 1/2) \Delta_\theta + i \Delta_\theta \qquad
    0 \le i < N_\theta-1
\end{displaymath}
where $N_\theta$ is the total number of intervals (\verb|sf_ntheta|),
$n_\theta$ is the number of ghost zones in the $\theta$ direction
(\verb|nghosttheta|) and $\Delta_\theta = \frac{\pi}{N_\theta-2n\theta}$.
Since with this 
\begin{displaymath}
    \theta_i \in [\Delta_\theta/2, \pi - \Delta_\theta/2] \qquad 
    \mbox{for $i \in [n_\theta, N_\theta-n_\theta-1]$}
\end{displaymath}
we do not have grid points at the end of the interval $\{0,\pi\}$ we derive
an open extended Newton-Cotes formula from Eq.~4.1.14 of~\cite{nr-cpp-2nd} and
a third order accurate extrapolative rule (see Maple worksheet). We find
\begin{eqnarray}
    \int_{x_0}^{x_{N-1}} f(x) dx &\approx& h \Bigl\{ 
    \frac{13}{12} f_{1/2} + \frac{7}{8} f_{3/2} + \frac{25}{24} f_{5/2} +
    f_{7/2} + f_{9/2} + \cdots + f_{N-1-7/2} \nonumber \\
    && + f_{N-1-9/2}
    + \frac{25}{24} f_{N-1-5/2} + \frac{7}{8} f_{N-1-3/2} + \frac{13}{12} f_{N-1-1/2}
    \Bigr\} + O(1/N^4)
\end{eqnarray}.

For the
$\phi$ direction SphericalSurfaces defines grid points such that
\begin{displaymath}
    \phi_i = -n_\phi \Delta_\phi + i \Delta_\phi \qquad
    0 \le i < N_\phi-1
\end{displaymath}
where $N_\phi$ is the total number of intervals (\verb|sf_nphi|),
$n_\phi$ is the number of ghost zones in the $\phi$ direction
(\verb|nghostphi|) and $\Delta_\phi = \frac{\pi}{N_\phi-2n\phi}$.
With this 
\begin{displaymath}
    \phi_i \in [0, 2 \pi - \Delta_\phi] \qquad 
    \mbox{for $i \in [n_\phi, N_\phi-n_\phi-1]$}
\end{displaymath}
we use a simple extended trapezoid rule to achieve spectral convergence due to
the periodic nature of $\phi$ (note: $x_N = x_0$)
\begin{eqnarray}
    \int_{x_0}^{x_N} f(x) dx  \approx& h \sum_{i=0}^{N-1} f_i
\end{eqnarray}.

The derivatives of $\vec{\bar r}$ along $\theta$ and $\phi$ are obtained
numerically and require at least two ghost zones in $\theta$ and $\phi$.

\section{Using This Thorn}
Right now surface can only be prescribed by ShericalSurfaces, the flux through
each surface is output in in a file \verb|outflow_det_%d.asc|

\subsection{Interaction With Other Thorns}
Takes care to schedule itself after \verb|SphericalSurfaces_HasBeenSet|.

\subsection{Examples}
See the parameter file in the test directory. For spherical symmetric infall
the flux through all detectors should be equal (since rest mass must be
conserved).

\begin{thebibliography}{9}
    \bibitem{lrr-font} Jos\'e A. Font,
    ``Numerical Hydrodynamics and Magnetohydrodynamics in General Relativity'',
    Living Rev. Relativity 11,  (2008),  7. URL (cited on August 15. 2009):
    http://www.livingreviews.org/lrr-2008-7

    \bibitem{nr-cpp-2nd} Press, W. H. (2002). Numerical recipes in C++: the
        art of scientific computing. Array Cambridge, UK: Cambridge University
        Press. 
\end{thebibliography}

% Do not delete next line
% END CACTUS THORNGUIDE

\end{document}
